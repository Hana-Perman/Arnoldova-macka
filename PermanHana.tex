\documentclass[GmVic, tisk]{gv}
% \documentclass[fin1, tisk]{fmfdelo}
% Če pobrišete možnost tisk, bodo povezave obarvane,
% na začetku pa ne bo praznih strani po naslovu, …

%%%%%%%%%%%%%%%%%%%%%%%%%%%%%%%%%%%%%%%%%%%%%%%%%%%%%%%%%%%%%%%%%%%%%%%%%%%%%%%
% METAPODATKI
%%%%%%%%%%%%%%%%%%%%%%%%%%%%%%%%%%%%%%%%%%%%%%%%%%%%%%%%%%%%%%%%%%%%%%%%%%%%%%%

% - vaše ime
\avtor{Ime Priimek}

% - naslov dela v slovenščini
\naslov{Naslov dela}

% - naslov dela v angleščini
\title{Angleški prevod slovenskega naslova dela}

% - ime mentorja/mentorice s polnim nazivom:
%   - doc.~dr.~Ime Priimek
%   - izr.~prof.~dr.~Ime Priimek
%   - prof.~dr.~Ime Priimek
%   za druge variante uporabite ustrezne ukaze
\mentorica{izr.~prof.~dr.~Ime Priimek}
\somentor{doc.~dr.~Ime Priimek}

% - leto diplome
\letnica{2016}

% - povzetek v slovenščini
%   V povzetku na kratko opišite vsebinske rezultate dela. Sem ne sodi razlaga
%   organizacije dela, torej v katerem razdelku je kaj, pač pa le opis vsebine.
\povzetek{V povzetku na kratko opišemo vsebinske rezultate dela. Sem ne sodi
razlaga organizacije dela -- v katerem poglavju/razdelku je kaj, pač pa le opis
vsebine.}

% - povzetek v angleščini
\abstract{Prevod slovenskega povzetka v angleščino.}

% - klasifikacijske oznake, ločene z vejicami
%   Oznake, ki opisujejo področje dela, so dostopne na strani https://www.ams.org/msc/
\klasifikacija{74B05, 65N99}

% - ključne besede, ki nastopajo v delu, ločene s \sep
\kljucnebesede{naravni logaritem\sep nenaravni algoritem}

% - angleški prevod ključnih besed
\keywords{natural logarithm\sep unnatural algorithm} % angleški prevod ključnih besed

% - angleško-slovenski slovar strokovnih izrazov
\slovar{
\geslo{continuous}{zvezen}
\geslo{uniformly continuous}{enakomerno zvezen}
\geslo{compact}{kompakten -- metrični prostor je kompakten, če ima v njem vsako zaporedje stekališče; podmnožica evklidskega prostora je kompaktna natanko tedaj, ko je omejena in zaprta  }
\geslo{glide reflection}{zrcalni zdrs ali zrcalni pomik -- tip ravninske evklidske izometrije, ki je kompozitum zrcaljenja in translacije vzdolž iste premice}
\geslo{lattice}{mreža}
\geslo{link}{splet}
\geslo{partition}{\textbf{$\sim$ of a set} razdelitev množice; \textbf{$\sim$ of a number} razčlenitev števila}
}

% - ime datoteke z viri (vključno s končnico .bib), če uporabljate BibTeX
\literatura{literatura.bib}

%%%%%%%%%%%%%%%%%%%%%%%%%%%%%%%%%%%%%%%%%%%%%%%%%%%%%%%%%%%%%%%%%%%%%%%%%%%%%%%
% DODATNE DEFINICIJE
%%%%%%%%%%%%%%%%%%%%%%%%%%%%%%%%%%%%%%%%%%%%%%%%%%%%%%%%%%%%%%%%%%%%%%%%%%%%%%%

% naložite dodatne pakete, ki jih potrebujete
\usepackage{algpseudocode}  % za psevdokodo
\usepackage{algorithm}      % za algoritme
\floatname{algorithm}{Algoritem}
\renewcommand{\listalgorithmname}{Kazalo algoritmov}

% deklarirajte vse matematične operatorje, da jih bo LaTeX pravilno stavil
% \DeclareMathOperator{\conv}{conv}
% na razpolago so naslednja matematična okolja, ki jih kličemo s parom
% \begin{imeokolja}[morebitni komentar v oklepaju] ... \end{imeokolja}
%
% definicija, opomba, primer, zgled, lema, trditev, izrek, posledica, dokaz

% za številske množice uporabite naslednje simbole
\newcommand{\R}{\mathbb R}
\newcommand{\N}{\mathbb N}
\newcommand{\Z}{\mathbb Z}
% Lahko se zgodi, da je ukaz \C definiral že paket hyperref,
% zato dobite napako: Command \C already defined.
% V tem primeru namesto ukaza \newcommand uporabite \renewcommand
\newcommand{\C}{\mathbb C}
\newcommand{\Q}{\mathbb Q}
%\newcommand{\det}{\mathrm{det}}
%%%%%%%%%%%%%%%%%%%%%%%%%%%%%%%%%%%%%%%%%%%%%%%%%%%%%%%%%%%%%%%%%%%%%%%%%%%%%%%
% ZAČETEK VSEBINE
%%%%%%%%%%%%%%%%%%%%%%%%%%%%%%%%%%%%%%%%%%%%%%%%%%%%%%%%%%%%%%%%%%%%%%%%%%%%%%%

\begin{document}

\section{Uvod}

Na začetku prvega poglavja/razdelka (ali v samostojnem razdelku z naslovom
Uvod) napišite kratek zgodovinski in matematični uvod. Pojasnite motivacijo za
problem, kje nastopa, kje vse je bil obravnavan. Na koncu opišite tudi
organizacijo dela -- kaj je v kakšnem razdelku.

Če se uvod naravno nadaljuje v besedilo prvega poglavja, lahko nadaljujete z
besedilom v istem razdelku, sicer začnete novega. Na začetku vsakega
razdelka/podraz\-delka poveste, čemu se bomo posvetili v nadaljevanju. Pri
pisanju uporabljajte ukaze za matematična okolja, med formalnimi enotami
dodajte vezno razlagalno besedilo.

\section{Matrike in determinante}
\begin{definicija} Realna ${2\times 2}$ matrika $M$  je ${2\times 2}$ shema realnih števil
$$M=
     \begin{bmatrix}
         a&b\\
         c&d
      \end{bmatrix}.
$$
Množico vseh realnih ${2\times 2}$ matrik označimo z $\R^{2\times 2}$
${2\times 2}$ matrike lahko med sabo seštevamo in množimo ter jih množimo s skalarjem po naslednjih pravilih:
Naj bosta         
$$A=
     \begin{bmatrix}
         a&b\\
         c&d
      \end{bmatrix},  \quad
  B=
     \begin{bmatrix}
         e&f\\
         g&h      \end{bmatrix} \mbox{ in } \quad r\in \R.
$$
Potem je 
$$
A + B = \begin{bmatrix}
         a+e&b+f\\
         c+g&d+h
      \end{bmatrix}, \quad 
rA= \begin{bmatrix}
         ra&rb\\
         rc&rd
      \end{bmatrix} \quad \mbox{ in } \quad
AB=  \begin{bmatrix}
         ae+bg&af+bh\\
         ce+dg&cf+dh
      \end{bmatrix}.
$$
Na matrikah definiramo tudi operacijo transponiranje:
 $$A^{T}=
     \begin{bmatrix}
         a&c\\
         b&d
      \end{bmatrix}. 
$$ 
\end{definicija}
Opomnimo, da množenje matrik ni komutativno. Nalednje trditve opisujejo lastnosti seštevanja, množenja in transponiranja matrik.

\begin {trditev}
Velja naslednje:
\begin{enumerate}
\item[1.] Komutativnost seštevanja: $A+B=B+A$,
\item[2.] Distributivnost: $A(B+C)= AB +AC$ in $(A+B)C=AC+BC$,
\item[3.] Asociativnost množenja: $A(BC)=(AB)C$ in
\item[4.] $(AB)^{T}=  B^T A^T$.
\end{enumerate}
\end{trditev}

\begin{trditev}
Realne $2 \times 2 $ matrike so grupa za seštevanje. Nevtralni element je ničelna matrika 
$$0=
     \begin{bmatrix}
         0&0\\
         0&0
      \end{bmatrix}, 
$$
naprotna matrika matriki $A$ pa je matrika $-A$.
\end{trditev}

\begin{definicija}
Matrka $I$ je nevtralni element za množenje ali enota za množenje je, imenujemo jo {\em identiteta} in je enaka
$$ I=
     \begin{bmatrix}
         1&0\\
         0&1
      \end{bmatrix}.
$$

{\em Diagonalna matrika} $D$ ima neničlne elemente samo na glavni diagonalni.
$$ D=
     \begin{bmatrix}
         d_1&0\\
         0&d_2
      \end{bmatrix}.
$$

Matrika $A$ je {\em obrnljiva}, če obstaja taka matrika $X$, da je $AX=XA=I$. Matriko $X$ imenujemo {\em inverzna} matrika matrike $A$.
\end{definicija}

\begin{trditev} Obrnljive matrike so grupa za množenje, kjer je matrika $I$ identiteta, inverzni element elementa pa je njegova inevrzna matrika.
\end{trditev}

%\begin{trditev} Za matriko $A$ in njeno inverzno matriko $X$ velja:
%$$
%X= \frac{1}{\det A} 
%\begin{bmatrix}
 %        d&-b\\
  %       -c&a
    %  \end{bmatrix}.
%$$
%\end{trditev}

%\begin{dokaz} To lahko računsko dokažem:
%\end{dokaz}

\begin{definicija} Naj bo        
$$A=
     \begin{bmatrix}
         a&b\\
         c&d
      \end{bmatrix}.
$$
{\em Determinanta} matrike $A$ je definirana kot $$\det A =ad - bc.$$
\end{definicija}

\begin{trditev}\label{trditev o determinantah} Naj bosta $A, B \in \R^{2\times 2}$. Potem velja $$\det(AB)= \det(A)\det(B) \mbox{ in } \det(A)^T=\det(A).$$
\end{trditev}

\begin{dokaz} To lahko računsko dokažem. Naj bosta
$$A=
     \begin{bmatrix}
         a&b\\
         c&d
      \end{bmatrix},  \quad
  B=
     \begin{bmatrix}
         e&f\\
         g&h      \end{bmatrix} \mbox{ in } \quad r\in \R.
$$
Dokazujemo, da  velja $\det(AB)=\det(A)\det(B).$
\begin{eqnarray*}
&&\det \begin{bmatrix}
         ae+bg&af+bh\\
         ce+dg&cf+dh
      \end{bmatrix} =
\det 
     \begin{bmatrix}
         a&b\\
         c&d
      \end{bmatrix}
  \det
     \begin{bmatrix}
         e&f\\
         g&h      
	\end{bmatrix}\\
&&(ae+bg)(cf+dh)-(af+bh)(ce+dg)= (ad-bc)(eh-fg)\\
&&acef+adeh+bcfg+bdgh-acef-adfg-bceh-bdgh = \\
&& \phantom{hancka}= adeh-adfg-bceh+bcfg \\
&&acef-acef+adeh+bcfg+bdgh-bdgh -adfg-bceh =\\
&&\phantom{hancka}=adeh+bcfg -adfg-bceh\\
&&adeh+bcfg-adfg-bceh=adeh+bcfg-adfg-bceh.
\end{eqnarray*}
\end{dokaz}

\begin{trditev} Naj bo $A \in \R^{2\times 2}$. Matrika $A$ obrnjljiva natanko tedaj, ko je  $\det A \ne  0$.
\end{trditev}

\begin{dokaz} Naj bo $A$ obrnljiva in $X$ njena inverzna matrika. Po trditvi \ref{trditev o determinantah} $AX=I$, torej je tudi $\det (AX)= \det A \det X =\det I= 1$, kar pomeni, da je $\det A$ različna od $0.$ Dokažimo še obratno. Če je $\det A \ne  0$, je inverzna matrika dana s formulo 
$$X=A^{-1}=\frac{1}{\det A} 
     \begin{bmatrix}
         d&-b\\
         -c&a
      \end{bmatrix}.
$$
\end{dokaz}

\begin{opomba} Če je $A$ celoštevilska matrika z determinanto $1$ ali $-1$ je tudi inverzna matrika celoštevilska.
\end{opomba}

\section{Linearne preslikave v $\R^2$ in matrike}

\begin{definicija}Z $\R^2$ označimo množico vseh vektorjev v ravnini, 
\[
\R^2=\left\{ \begin{bmatrix}
         a\\
         b
      \end{bmatrix}, a,b \in \R^2\right\}.
\]
\end{definicija}

\begin{definicija}
Preslikava $\mathcal{A}: \R^2\rightarrow\R^2$ je {\em linearna}, če za vsaka dva vektorja $\vec{v}_1, \vec{v}_2 \in \R^2$ in za vsaki realni števili $\lambda_1, \lambda_2$ velja 
\[
\mathcal{A}(\lambda_1\vec{v}_1+\lambda_2 \vec{v}_2)=\lambda_1\mathcal{A}(\vec{v}_1)+\lambda_2\mathcal{A}(\vec{v}_2).
\]
Zapisu $\lambda_1\vec{v}_1+\lambda_2 \vec{v}_2$ pravimo {linearna kombinacija}.
\end{definicija}

\begin{definicija} Množica $\{\vec{v}_1, \vec{v}_2 \}$ je {\em baza}, če lahko vsak vektor $\vec{v} \in \R^2$ na en sam način zapišemo kot \[
\vec{v}=\lambda_1\vec{v}_1+\lambda_2 \vec{v}_2.
\]
\end {definicija}

\begin{trditev} Množica $\{\vec{v}_1, \vec{v}_2 \}$ je baza natanko tedaj ko sta  $\vec{v}_1, \vec{v}_2 $  nevzporedna. (dokaži!!)
\end{trditev}

\begin{definicija} Množica 
$$\left\{ \vec e_1=\begin{bmatrix}
         1\\
         0
      \end{bmatrix},
\vec e_2=\begin{bmatrix}
         0\\
         1
      \end{bmatrix}\right\}
$$
 je {\em standardna baza}.
\end {definicija}

\begin{trditev}Vektorja 
$$\vec{v}_1=
 \begin{bmatrix}
         a\\
         b
      \end{bmatrix},
 \vec{v}_2 = \begin{bmatrix}
         c\\
         d
      \end{bmatrix},$$ 
sta nevzporedna natanko tedaj, ko 
 $$\det \begin{bmatrix}
         a&b\\
         c&d
      \end{bmatrix}\ne 0.$$ 
\end{trditev}
%dokaži sama

\begin{trditev}Linearna preslikava je enolično določena s slikama dveh nevzporednih vektorjev.
\end{trditev}
%dokaži sama samcata

Poglejmo kako lahko linearno preslikavo $\mathcal A$ zapišemo v bazi  $\{\vec{v}_1, \vec{v}_2 \}$. Če je 
\begin{equation}\label{enačba o preslikavi A}
\vec v =x_1 \vec{v}_1 + x_2\vec{v}_2, \quad \mathcal A(\vec v_1)=a\vec{v}_1+ c \vec{v}_2\, \mbox{  in  }\,
\mathcal A(\vec v_2)=b\vec{v}_1+ d \vec{v}_2 ,
\end{equation}
potem je 
\begin{eqnarray*}
\mathcal A (\vec v)&=& x_1\mathcal A(\vec v_1) + x_2 \mathcal A(\vec v_2)\\
			&=& x_1(a\vec{v}_1+ c \vec{v}_2) + x_2(b\vec{v}_1+ d \vec{v}_2) \\
			&=& \vec{v}_1(x_1a+x_2b) + \vec{v}_2 (x_1c+x_2d)\\
			&=& y_1 \vec{v}_1 + y_2 \vec{v}_2.
\end{eqnarray*}
Dogovorimo se, da bomo vsak vektor $\vec v= x_1 \vec{v}_1 + x_2\vec{v}_2$ identificirali z vektorjem komponent v baznih smereh. Zato je
  $$\vec v_1 \simeq \begin{bmatrix}
         1\\
         0
      \end{bmatrix}, \, 
\vec v_2 \simeq \begin{bmatrix}
         0\\
         1
      \end{bmatrix},\, 
\vec v \simeq \begin{bmatrix}
         x_1\\
         x_2
      \end{bmatrix},
\mathcal A(\vec v) \simeq \begin{bmatrix}
         y_1\\
         y_2
      \end{bmatrix}, 
$$
%
Če sedaj linearni preslikavi $\mathcal A$ priredimo matriko 
$$A=
     \begin{bmatrix}
         a&b\\
         c&d
      \end{bmatrix}, $$
dobimo zvezo 
$$\begin{bmatrix}
         y_1\\
         y_2
      \end{bmatrix} =
 \begin{bmatrix}
         a&b\\
         c&d
      \end{bmatrix} 
 \begin{bmatrix}
         x_1\\
         x_2
      \end{bmatrix}.
$$
Linearni preslikavi $\mathcal A$ v bazi $\{\vec v_1,\vec v_2\}$ pripada matrika $A$. V drugih bazah isti linearni preslikavi pripada drugačna matrika.

\begin{trditev}Če imamo linearno preslikavo $\mathcal A$, ki ji v bazi $\{\vec v_1,\vec v_2\}$ pripada matrika $A$ in linearno preslikavo $\mathcal B$, ki ji v bazi $\{\vec v_1,\vec v_2\}$ pripada matrika $B$, potem preslikavi  $\mathcal A \circ \mathcal B$ v isti bazi pripada matrika $AB$.
%dokaži sama
\end{trditev}

\begin{trditev} Izberimo bazo $\{\vec v_1,\vec v_2\}$ in naj linearni preslikavi $\mathcal A$ pripada matrika $A$, potem linearna preslikava $\mathcal A$ ohranja ploščino natanko tedaj, ko je $\det A = \pm 1$.
%dokaži sama  
\end{trditev}

\section{Lastne vrednosti in lastni vektorji}
Edine matrike, ki jih lahko enostvano množimo so diagonalne matrike. Diagonalna matrika:
$$
  A=\begin{bmatrix}
         a&0\\
         0&b
      \end{bmatrix}, 
  B=\begin{bmatrix}
         c&0\\
         0&d
      \end{bmatrix}, 
AB=  \begin{bmatrix}
         ac&0\\
         0&bd
      \end{bmatrix}.
$$
Če želimo izračunati večkratni kompozitum linearne preslikave $\mathcal A$ s samo sabo, je zato smiselno zapisati linearno preslikavo v taki bazi, da ji bo pripadala diagonalna matrika (če je to mogoče). 	Denimo da baza $\{\vec v_1, \vec v_2\}$ je taka baza in pripadajoča diagonalna matrika enaka 
$$ D=
     \begin{bmatrix}
         \lambda_1&0\\
         0&\lambda_2
      \end{bmatrix}.
$$
To pomeni, da mora veljati $\mathcal A(\vec v_1)= \lambda \vec v_1$ in $\mathcal A(\vec v_2)= \lambda \vec v_2.$ Poglejmo kdaj je enačba $\mathcal A(\vec v) = \lambda \vec v$ rešljiva.
Naj linearni preslikavi $\mathcal A$ v standradni bazi pripada matrika
$$ A=\begin{bmatrix}
         a&b\\
         c&d
      \end{bmatrix}$$
%
Naj bo vektor $\vec v$ enak
$$\vec v \simeq \begin{bmatrix}
         x_1\\
         x_2
      \end{bmatrix}
$$
Potem se enačba po komponentah prepiše kot
\begin{eqnarray*}
&&y_1= \mathcal A(x_1)=\lambda x_1, \quad y_2= \mathcal A(x_2)=\lambda x_2\\
&&y_1 = ax_1+bx_2 = \lambda x_1, \quad y_2 = cx_1+dx_2 = \lambda x_2\\
&&x_1(a-\lambda)+bx_2 =0, \quad cx_1+x_2(d - \lambda)=0
\end{eqnarray*}
Ko uredimo sistem, dobimo
\[x_1((a-\lambda)(d-\lambda)-bc)=0, \quad
x_2((a-\lambda)(d-\lambda)-bc)=0
\]
Če je $P(\lambda):=((a-\lambda)(d-\lambda)-bc) \ne 0$ sta $x_1,x_2=0$ in $\vec v$ ni bazni vektor, zato mora biti $P(\lambda)=0$.
Iz tega sledi, da:
$$
P(\lambda)=\lambda^2-(a+d)\lambda +ad -bc=0,  \quad
\lambda_{1,2}= \frac{a+d \pm \sqrt{(a+d)^2-4(ad-bc)}}{2}.
$$
Opazimo, da 
$$
\lambda_1+\lambda_2= a+d, \, \lambda_1\lambda_2= ad-bc=\det A \mbox{ in } P(\lambda)=\det(A-\lambda I).
$$
%
\begin{definicija} Naj bo $\mathcal A$ linearna preslikava. Neničelni vektor ${\vec v}$ je {\em lastni vektor} ${\mathcal A}$ če obstaja tako realno število $\lambda$, imenovano {\em lastna vrednost}, da velja 
$$
\mathcal A(\vec v)= \lambda \vec v.
$$
Par $(\lambda, \vec v)$ imenujemo {\em lastni par}, polinom $P(\lambda)= \det( A -\lambda I) $  pa {\em karakteristični polinom.}
\end{definicija}
%
Latni vrednosti sta ničli karakterističnega polinoma. Nas bodo zanimale linearne preslikave, ki ohranjajo ploščino in nimajo lastnih vrednosti z absolutno vrednostjo $1$.

\begin{definicija} {\em Hiperbolična linearna preslikava} je preslikava, ki ohranja plošino in  ima lastni vrednosti, ki ne ležita na enotski krožnici v kompleksni ravnini.
\end{definicija}

\begin{trditev} Hiperbolična linearna preslikava nima kompleksno konjugiranih lastnih vrednosti in ima dve realni lastni vrednosti $\lambda_1, \lambda_2$ s tem da je 
$$
0<|\lambda_1|<1 <|\lambda_2|.
$$
\end{trditev}







\end{document}


Podrobneje si poglejmo naslednji rezultat. Ker se bomo kasneje nanj sklicevali,
si pripravimo oznako z ukazom \verb|\label{oznaka}|. Seveda morajo biti oznake
različnih trditev različne. Enako označujemo tudi druga okolja oziroma enote
besedila.

\begin{izrek}\label{izr:enakomerno}
Zvezna funkcija na zaprtem intervalu je enakomerno zvezna.
\end{izrek}

\begin{dokaz}
Na začetku dokaza, če je to le mogoče in smiselno, razložite idejo dokaza.

Dokazovali bomo s protislovjem. Pomagali si bomo z definicijo zveznosti in s
kompaktnostjo intervala.  Izberimo $\varepsilon>0$. Če $f$ ni enakomerno
zvezna, potem za vsak $\delta>0$ obstajata $x, y$, ki zadoščata
\begin{equation}\label{eq:razlika}
  |x-y|<\delta\quad \text{in}\quad |f(x)-f(y)| \ge \varepsilon. \qedhere
\end{equation}
\end{dokaz}

V zgornjem primeru smo kvadratek za konec dokaza postavili v zadnjo vrstico
besedila, ki je vrstična formula, s pomočjo ukaza \verb|\qedhere|.  Ta ukaz
ustrezno deluje znotraj okolij \emph{equation, align*} in podobnih, ne pa
znotraj \verb|$$ ... $$|.

Oglejmo si še enkrat neenačbi~\eqref{eq:razlika}. Na formule se sklicujemo z
ukazom \verb|\eqref{oznaka}|, ki postavi zaporedno številko enačbe
v oklepaje, na trditve in ostale enote pa z ukazom \verb|\ref{oznaka}|. Črni
pravokotnik ob robu strani označuje predolgo vrstico, kjer \LaTeX ni uspel
pravilno postaviti besedila, zato mu morate pomagati, npr.\ tako, da stavek
nekoliko preoblikujete, sami razdelite nedeljivo enoto (npr. razdelite
matematično formulo na dva dela) ali pa ponudite možnosti za deljenje težavne
besede s pomočjo znakov\verb|\-|, ki jih postavite na mesto, kjer se besedo sme
deliti, npr.\  \verb|te\-žav\-nost\-ni\-ca|. Zgoraj bi tako lahko zapisali:

Oglejmo si še enkrat neenačbi~\eqref{eq:razlika}. Za sklicevanje na označene
enote besedila imamo na razpolago dva ukaza; na formule se sklicujemo z ukazom
\verb|\eqref{oznaka}|, \dots

V predzadnjem odstavku je v oklepaju za okrajšavo npr.\ nastal predolg
presledek sredi stavka, saj je \LaTeX zaradi pike sklepal, da je na tem mestu
stavka konec. Tak predolg presledek preprečimo tako, da za piko sredi stavka
postavimo poševnico in za njo presledek, torej \verb|\ |.\\

Če dokaz trditve ne sledi neposredno formulaciji trditve, moramo povedati, kaj
bomo dokazovali. To naredimo tako, da ob ukazu za okolje dokaz dodamo neobvezni
parameter,  v katerem napišemo besedilo, ki se bo izpisalo namesto besede
\emph{Dokaz}, npr.\ \verb|\begin{Dokaz}[Dokaz izreka \ref{izr:enakomerno}]|.

\begin{dokaz}[Dokaz izreka \ref{izr:enakomerno}]
  Dokazovanja te trditve se lahko lotimo tudi takole \ldots
\end{dokaz}

\subsection{Naslov morebitnega (pod)razdelka} Besedilo naj se nadaljuje v vrstici naslova, torej za ukazom \verb|\subsection{}| ne smete izpustiti prazne vrstice.

Podobno kot lahko spremenimo ime dokaza, lahko dodamo komentar v ime trditve,
torej s pomočjo neobveznega parametra; prvega od spodnjih izpisov dosežemo z
ukazom \verb|\begin{posledica}[izrek o vmesni vrednosti]|. Če želimo v
parametru navesti vir, pri katerem bomo navedli podatek o tem, kje v viru to
trditev najdemo, pa uporabimo ukaz
\verb|\begin{posledica}[\protect{\cite[izrek 3.14]{glob}}]|. Seveda lahko obe
možnosti kombiniramo.


\begin{posledica}[izrek o vmesni vrednosti]
  Naj bo $f$ zvezna in \ldots
\end{posledica}

Ali pa

\begin{posledica}[izrek o vmesni vrednosti \protect{\cite[izrek 3.14]{glob}}]
  Naj bo $f$ zvezna in \ldots
\end{posledica}

Podobno lahko napovemo tudi vsebino primera.

\begin{primer}[nezvezna funkcija nima nujno lastnosti povprečne vrednosti]
  Naj bo $f \colon \R \to \R$ dana s predpisom \dots
\end{primer}

\subsection{Pisanje algoritmov}
Za pisanje algoritmov sta na voljo okolji \texttt{algorithm} in
\texttt{algorithmic} iz paketov \texttt{algorithm} in \texttt{algorithmix}, ki
sodelujeta podobno kot \texttt{table} in \texttt{tabular}. Algoritmi plavajo
med tekstom, enako kot slike in tabele, nanje se lahko tudi sklicujemo, kot
prikazano v izvorni kodi in v algoritmu~\ref{alg:metoda}. Sklicujemo se lahko
tudi na pomembne vrstice, npr.\ na vrstico~\ref{alg:pomembna-vrstica}, ki
predstavlja glavni del algoritma. Za primer pisanja algoritma se posvetujte s
primerom v tem dokumentu, za bolj napredne primere uporabe, kot na primer
razbijanje algoritma na več kosov, pa z (precej razumljivo) uradno
dokumentacijo\footnote{\url{http://tug.ctan.org/macros/latex/contrib/algorithmicx/algorithmicx.pdf}}.
Če želite vključiti izvorno kodo nekega programa, priporočamo paket
\texttt{minted}\footnote{\url{https://github.com/gpoore/minted}}.

\algnewcommand\algorithmicto{\textbf{to}}
\algnewcommand\algorithmicin{\textbf{in}}
\algnewcommand\algorithmicforeach{\textbf{for each}}
\algrenewtext{For}[3]{\algorithmicfor\ #1 $\gets$ #2\ \algorithmicto\ #3\ \algorithmicdo}
\algdef{S}[FOR]{ForEach}[2]{\algorithmicforeach\ #1\ \algorithmicin\ #2\ \algorithmicdo}

\begin{algorithm}[ht]
  \caption{Opis, ki ima enako funkcionalnost kot opis pod sliko.}
  \label{alg:metoda}
  \raggedright
  \textbf{Vhod:} Števili $n, m \in \N, n > m$. \\
  \textbf{Izhod:} Decimalno število $x$, ki aproksimira rešitev enačbe $n x = m$.
  \begin{algorithmic}[1]
    \Function{reši}{$n$, $m$} \Comment{Vsi vhodni parametri morajo biti opisani.}
    \State $a \gets [\,]$ \Comment{Spremenljivka $a$ naj postane prazna kopica.}
    \For{$i$}{$1$}{$n$}
      \If{$i \operatorname{mod} 7 = 5$}
        \State \Call{heapop}{$a$}
      \ElsIf{$i < 5$}
      \State \Call{heappush}{$a, \frac{i+12}{7} + \pi$} \Comment{Lahko uporabljamo matematiko.}
      \Else
        \State \Call{heappush}{$a, i$}
      \EndIf
    \EndFor
    \Statex  \Comment{Prazna vrstica}
    \State $x \gets 0$  \Comment{To je primer komentarja.}
    \ForEach{e}{a}
      \State $x \gets 1 + \sqrt[e]{x}$
    \EndFor
    \While{$|x| > \varepsilon$}
      \State $x \gets x / 2$
    \EndWhile
    \State $x \gets m / n$ \label{alg:pomembna-vrstica}
    \State \Return $x$  \Comment{Vsi izhodni parametri morajo biti opisani nad algoritmom.}
    \EndFunction
  \end{algorithmic}
\end{algorithm}

\subsection{Stavljenje slik}

Če je slika samo ena, jo vstavimo z ukazi (pobrisati \verb|\verb{...} | iz kode)

\verb|\begin{figure}[h]|

\verb|\includegraphics[width=0.4\textwidth]{TAnn.pdf}|

\verb|\caption{Symmetric tiling for the annulus $\cl(A(0;1,3))$ with $c_0 = 1,c_1 =2$ and $c_2 = 3.$}\label{TFE}|

\verb|\end{figure}|

Label pove, da se bo v spremenljivko TFE skhranila številka slike. Do te številke pridemo z ukazom \verb|\ref|.

Slike v tabeli. Tri slike, Tiling.pdf, LadyBug1.pdf in LadyBugTiled1.pdf so v tabeli slik, prav tako nihove oznake (a),(b),(c). Spodaj je opis slik - vidno v kodi.
%\begin{figure}[h]
% \centering
% $\begin{array}{ccccc}
%  {\includegraphics[width=0.3\textwidth]{IntTiling.pdf}}&
%  {\includegraphics[width=0.3\textwidth]{LadyBug1.pdf}}&
%  {\includegraphics[width=0.3\textwidth]{LadyBugTiled1.pdf}} \\
%  (a)&(b)&(c)
% \end{array}$
%  \caption{(a) the set $B \cap K_{n+1, \nu}$, (b) the set $K_{n}\cap B$  (blue), already tiled, with a hole $D$ (light blue centered in the origin) that intersects the real axis and with a pair of symmetric holes (light blue), filled components of $K_n$ (dark blue), domain to be tiled (light blue) and holes of $K_{n+1}$ (white), (c) the symmetric tiling of the symmetric necklace.}\label{case2b}
%\end{figure}


\section{Konec dela}

Na konec dela sodita angleško-slovenski slovarček strokovnih izrazov in seznam
uporabljene literature, morebitne priloge (programska koda, daljša ponovitev
dela snovi, ki je bil obravnavan med študijem \dots) pa neposredno pred ti
enoti. Slovar naj vsebuje vse pojme, ki ste jih spoznali ob pripravi dela, pa
tudi že znane pojme, ki ste jih spoznali pri izbirnih predmetih. Najprej
navedite angleški pojem (ti naj bodo urejeni po abecedi) in potem ustrezni
slovenski prevod; zaželeno je, da temu sledi tudi opis pojma, lahko komentar
ali pojasnilo. Slovarska gesla navajajte z ukazom \verb|\geslo{}{}|, npr.\
\verb|\geslo{continuous}{zvezen}|.

Pri navajanju literature si pomagajte s spodnjimi primeri; najprej je opisano
pravilo za vsak tip vira, nato so podani primeri. Člen literature napišete z
ukazom \verb|\bibitem{oznaka} podatki o viru|, kjer mora \emph{oznaka} enolično
določati vir.  Posebej opozarjam, da spletni viri uporabljajo paket url, ki je
vključen v~.cls datoteki. Polje ``ogled'' pri spletnih virih je obvezno; če je
kak podatek neznan, ustrezno ``polje'' seveda izpustimo. Literaturo je potrebno
urediti po abecednem vrstnem redu; najprej navedemo vse vire z znanimi avtorji
(tiskane in spletne) po abecednem redu avtorjev (po priimkih, nato imenih),
nato pa spletne vire brez avtorjev, urejene po naslovih strani. Če isti vir
navajamo v dveh oblikah, kot tiskani in spletni vir, najprej navedemo tiskani
vir, nato pa še podatek o tem, kje je dostopen v elektronski obliki.

Citiramo z ukazom \verb|\cite{AB}|.

Janez Novak je napisal knjigo \cite{AB}.

\begin{thebibliography}{99}
\bibitem[VIR]{AB} Janez Novak, {\em Ta veseli dan}, Ljubljana 1789
\end{thebibliography}
\end{document}
